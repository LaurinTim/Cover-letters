\documentclass[10pt, a4paper]{article}

\input{cover-letter-themes/theme_selection}

%----------------------------------------------------------------------------------------
%	 THEMES
%----------------------------------------------------------------------------------------

% Define the desired theme out of the following: beige, blue, bw, coral, earth, framed, gray, minimal, onyx, plain
% See screenshots in preview/ directory
\theme{framed}

%----------------------------------------------------------------------------------------
%	 PERSONAL INFORMATION
%----------------------------------------------------------------------------------------

% If you don't need a particular field, just remove the content leaving the command, e.g. \jobtitle{}

\name{Laurin Koller} % Your name
\jobtitle{Praktikum in Aircraft Data Analytics} % Job title/career
\location{Bärengasse 3, 6317} % Address/location
\phone{079 587 16 59} % Phone number
\mail{laurin.koller@gmail.com} % Mail
\employerinfo{Federica Lionetto \\
Obstgartenstrasse 25 \\
Kloten, 8302 \\
044 564 00 00 \\
recruiting.services@swiss.com} % Contact information of the employer. Make sure to end every row with "\\"

\begin{document}

\makeprofile % Print name & job description

\makecontact % Print defined contact information

\today % Command for inserting today's date.

\letterspace % Command for adding larger vertical space
\makeemployerinfo % Print employer contact information

Sehr geehrte Frau Lionetto,

Fasziniert von der Schnittstelle zwischen Data Science und Luftfahrt, sehe ich in der ausgeschriebenen Position als Intern in Aircraft Data Analytics bei SWISS eine perfekte Gelegenheit, meine analytischen Fähigkeiten weiterzuentwickeln und insbesondere meine Kenntnisse im Bereich Natural Language Processing (NLP) auszubauen.

Während meiner Masterarbeit am CERN trainierte ich Machine Learning Modelle mit experimentellen Daten, um einzelne Partikel- und Photoneneinschläge zu erkennen. Zudem habe ich komplexe experimentelle Daten strukturiert und in verschiedene Teilsysteme integriert, ein Ansatz, der sich direkt auf die Analyse von Flugzeugdaten übertragen lässt. Meine fundierten Kenntnisse in Python, SQL und PySpark geben mir das Rüstzeug, um die Herausforderungen dieser Rolle aktiv mitzugestalten. Zudem verfüge ich über erste Erfahrung mit Cloud-Technologien, ein Bereich, den ich weiter vertiefen möchte.

Was mich besonders an SWISS reizt, ist die konsequente Digitalisierung und der Anspruch, das Flugerlebnis durch smarte Datenanalyse zu optimieren. Ich sehe in dieser Position die Chance, meine analytischen und technischen Fähigkeiten einzusetzen, während ich gleichzeitig meine Kenntnisse im Bereich NLP erweitere, um innovative Algorithmen zur Erkennung von Trends und Anomalien in Flugzeugdaten zu entwickeln.

Ich freue mich darauf, meine Stärken in Ihr Team einzubringen und SWISS aktiv in zukunftsweisenden Projekten zu unterstützen. Gerne stehe ich Ihnen für ein persönliches Gespräch zur Verfügung.

\letterspace
Beste Grüsse,

\letterspace
Laurin Koller

\end{document}
